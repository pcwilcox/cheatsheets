% LaTeX template adapted from: https://www.overleaf.com/latex/templates/simple-math-homework-template/tbszsswsndrz
\documentclass[landscape,8pt]{extarticle}
\usepackage[utf8]{inputenc}
\usepackage[english]{babel}
\usepackage{extsizes}
\usepackage[]{amsthm} %lets us use \begin{proof}
\usepackage[]{amssymb} %gives us the character \varnothing
\usepackage{amsmath} %for equations
\usepackage[]{listings} %for code blocks
\usepackage{graphicx} %for diagrams
\usepackage{fancyhdr} %for headers
\usepackage[letterpaper, margin=0.25in, headheight=10pt, headsep=0pt]{geometry}
\usepackage{tikz} % for drawings
\usepackage{multicol}
\usepackage{ifthen}
\usepackage{enumitem}
\usepackage{listings}
\usepackage{color}
\usepackage{adjustbox}

\usepackage{wrapfig}
\usepackage{booktabs}


\setlist{nolistsep}
\setlist[itemize]{leftmargin=0.25pc,itemsep=0em}
\usetikzlibrary{arrows.meta,shapes.arrows,chains,decorations.pathreplacing}
\makeatletter
\renewcommand{\section}{\@startsection{section}{1}{0mm}%
            {-1ex plus -.5ex minus -.2ex}%
            {0.5ex plus .2ex}%x
            {\normalfont\large\bfseries}}
\renewcommand{\subsection}{\@startsection{subsection}{2}{0mm}%
            {-1explus -.5ex minus -.2ex}%
            {0.5ex plus .2ex}%
            {\normalfont\normalsize\bfseries}}
\renewcommand{\subsubsection}{\@startsection{subsubsection}{3}{0mm}%
            {-1ex plus -.5ex minus -.2ex}%
            {1ex plus .2ex}%
            {\normalfont\small\bfseries}}
\makeatother
\definecolor{dkgreen}{rgb}{0,0.6,0}
\definecolor{gray}{rgb}{0.5,0.5,0.5}
\definecolor{mauve}{rgb}{0.58,0,0.82}

\lstset{
  frame=none,
  xleftmargin=2pt,
  stepnumber=1,
  numbers=none,
  numbersep=5pt,
  numberstyle=\ttfamily\tiny\color[gray]{0.3},
  belowcaptionskip=\bigskipamount,
  captionpos=b,
  escapeinside={*'}{'*},
  language=haskell,
  tabsize=2,
  emphstyle={\bf},
  commentstyle=\it,
  stringstyle=\mdseries\rmfamily,
  showspaces=false,
  keywordstyle=\bfseries\rmfamily,
  columns=flexible,
  basicstyle=\small\sffamily,
  showstringspaces=false,
  morecomment=[l]\%,
}

\newcommand{\code}{\lstinline}
\graphicspath{}
\pagestyle{empty}
\ifthenelse{\lengthtest{ \paperwidth= 11in}}
{ \geometry{top=.3in,left=.25in,right=.25in,bottom=.25in} }
{\ifthenelse{ \lengthtest{ \paperwidth= 297mm}}
{\geometry{top=1cm,left=1cm,right=1cm,bottom=1cm} }
{\geometry{top=1cm,left=1cm,right=1cm,bottom=1cm} }
}
\setlength{\parindent}{0em}
\setlength{\parskip}{-0.25em}
\pagestyle{fancy}
\fancyhf{}
\renewcommand{\headrulewidth}{0pt}
\rhead{Pete Wilcox | CruzID:\@ pcwilcox | Student ID:\@ 1593715}

\begin{document}
\footnotesize
\begin{multicols}{4}
    \setlength{\premulticols}{1pt}
    \setlength{\postmulticols}{1pt}
    \setlength{\multicolsep}{1pt}
    \setlength{\columnsep}{2pt}
    \begin{itemize}
        \item \emph{Syntax}
        \begin{lstlisting}
e ::= x
    | \x -> e
    | e1 e2
        \end{lstlisting}
        \begin{itemize}
            \item Programs are \emph{expressions} or $\lambda$-terms
            \item \emph{Variable:} \code{x}, \code{y}, \code{z}
            \item \emph{Abstraction:} (aka nameless function definition) \code{\x -> e} means ``for any \code{x}, compute \code{e}''; \code{x} is the \emph{formal parameter}, \code{e} is the \emph{body}
            \item \emph{Application:} (aka function call) \code{e1 e2} means ``apply \code{e1} to \code{e2}''; \code{e1} is the \emph{function} and \code{e2} is the \code{argument}
        \end{itemize}
        \item \emph{Syntactic Sugar:} convenient notation used as a shorthand for valid syntax
        \begin{lstlisting}
-- instead of:              we write:
\x -> (\y -> (\z -> e))     \x -> \y -> \z -> e
\x -> \y -> \z -> e         \x y z -> e
(((e1 e2) e3) e4)           e1 e2 e3 e4
        \end{lstlisting}
        \item \emph{Scope of a variable} The part of a program where a \emph{variable is visible}
        \item In the expression \code{\x -> e}
        \begin{itemize}
            \item \code{x} is the newly-introduced variable
            \item \code{e} is the \emph{scope} of \code{x}
            \item Any occurrence of \code{x} in \code{\x -> e} is \emph{bound} (by the \emph{binder} \code{\x})
        \end{itemize}
        \item \emph{Rewrite rules:}
        \begin{itemize}
            \item $\alpha$-step: renaming formals
            \item $\beta$-step: aka function call
        \end{itemize}
    \end{itemize}
    \begin{center}
        \begin{tabular}{ | c | c | } \toprule
            Var     & Desc                                      \\ \midrule
            $B$     & the number of data pages                  \\ \midrule
            $R$     & number of records per page                \\ \midrule
            $D$     & average time to read or write a disk page \\ \midrule
            $F$     & average fanout for a non-leaf page        \\ \bottomrule
        \end{tabular}
    \end{center}
    \begin{center}
        \begin{adjustbox}{angle=270}
            \begin{tabular}{ | c | c | c | c | c | c | } \toprule
                ()            & Scan          & Equality              & Range                                    & Insert        & Delete        \\ \midrule
                Heap          & $BD$          & $0.5 BD$              & $BD$                                     & $2D$          & Search + $D$  \\ \midrule
                Sorted        & $BD$          & $D \log_2 B$          & $D(\log_2 B +$ \# matching pages$)$      & Search + $BD$ & Search + $BD$ \\ \midrule
                Clustered     & $1.5 BD$      & $D \log_F 1.5 B$      & $D(\log_F 1.5B + $ \# matching pages$)$  & Search + $D$  & Search + $D$  \\ \midrule
                Unclust. Tree & $BD(R+0.15)$  & $D(1 + \log_F 0.15B)$ & $D(\log_F 0.15B + $ \# matching pages$)$ & Search + $2D$ & Search + $2D$ \\ \midrule
                Unclust. Hash & $BD(R+0.125)$ & $2D$                  & $BD$                                     & Search + $2D$ & Search + $2D$ \\ \bottomrule
            \end{tabular}
        \end{adjustbox}
    \end{center}
\end{multicols}

\end{document}
