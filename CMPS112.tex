% LaTeX template adapted from: https://www.overleaf.com/latex/templates/simple-math-homework-template/tbszsswsndrz
\documentclass[landscape,8pt]{extarticle}
\usepackage[utf8]{inputenc}
\usepackage[english]{babel}
\usepackage{extsizes}
\usepackage[]{amsthm} %lets us use \begin{proof}
\usepackage[]{amssymb} %gives us the character \varnothing
\usepackage{amsmath} %for equations
\usepackage[]{listings} %for code blocks
\usepackage{graphicx} %for diagrams
\usepackage{fancyhdr} %for headers
\usepackage[letterpaper, margin=0.25in, headheight=10pt, headsep=0pt]{geometry}
\usepackage{tikz} % for drawings
\usepackage{multicol}
\usepackage{ifthen}
\usepackage{enumitem}
\usepackage{listings}
\usepackage{color}
\usepackage{adjustbox}

\usepackage{wrapfig}
\usepackage{booktabs}


\setlist{nolistsep}
\setlist[itemize]{leftmargin=0.25pc,itemsep=0em}
\usetikzlibrary{arrows.meta,shapes.arrows,chains,decorations.pathreplacing}
\makeatletter
\renewcommand{\section}{\@startsection{section}{1}{0mm}%
            {-1ex plus -.5ex minus -.2ex}%
            {0.5ex plus .2ex}%x
            {\normalfont\large\bfseries}}
\renewcommand{\subsection}{\@startsection{subsection}{2}{0mm}%
            {-1explus -.5ex minus -.2ex}%
            {0.5ex plus .2ex}%
            {\normalfont\normalsize\bfseries}}
\renewcommand{\subsubsection}{\@startsection{subsubsection}{3}{0mm}%
            {-1ex plus -.5ex minus -.2ex}%
            {1ex plus .2ex}%
            {\normalfont\small\bfseries}}
\makeatother
\definecolor{dkgreen}{rgb}{0,0.6,0}
\definecolor{gray}{rgb}{0.5,0.5,0.5}
\definecolor{mauve}{rgb}{0.58,0,0.82}

\lstset{
  frame=none,
  xleftmargin=1pt,
  stepnumber=1,
  numbers=none,
  numbersep=5pt,
  numberstyle=\ttfamily\tiny\color[gray]{0.3},
  belowcaptionskip=\bigskipamount,
  captionpos=b,
  escapeinside={*'}{'*},
  language=haskell,
  tabsize=2,
  aboveskip=2mm,
  belowskip=2mm,
  emphstyle={\bf},
  commentstyle=\it,
  stringstyle=\mdseries\ttfamily,
  showspaces=false,
  keywordstyle=\bfseries\ttfamily,
  columns=flexible,
  basicstyle=\small\ttfamily,
  showstringspaces=false,
  morecomment=[l]\%,
}

\newcommand{\code}{\lstinline}
\graphicspath{}
\pagestyle{empty}
\ifthenelse{\lengthtest{ \paperwidth= 11in}}
{ \geometry{top=.3in,left=.25in,right=.25in,bottom=.25in} }
{\ifthenelse{ \lengthtest{ \paperwidth= 297mm}}
{\geometry{top=1cm,left=1cm,right=1cm,bottom=1cm} }
{\geometry{top=1cm,left=1cm,right=1cm,bottom=1cm} }
}
\setlength{\parindent}{0em}
\setlength{\parskip}{-0.25em}
\pagestyle{fancy}
\fancyhf{}
\renewcommand{\headrulewidth}{0pt}
\rhead{Pete Wilcox | CruzID:\@ pcwilcox | Student ID:\@ 1593715}

\begin{document}
\footnotesize
\begin{multicols}{3}
    \setlength{\premulticols}{1pt}
    \setlength{\postmulticols}{1pt}
    \setlength{\multicolsep}{1pt}
    \setlength{\columnsep}{2pt}
    \begin{itemize}
        \item \emph{Syntax}
              \begin{lstlisting}
e ::= x
    | \x -> e
    | e1 e2
        \end{lstlisting}
              \begin{itemize}
                  \item Programs are \emph{expressions} or $\lambda$-terms
                  \item \emph{Variable:} \code{x}, \code{y}, \code{z}
                  \item \emph{Abstraction:} (aka nameless function definition) \code{\x -> e} means ``for any \code{x}, compute \code{e}''; \code{x} is the \emph{formal parameter}, \code{e} is the \emph{body}
                  \item \emph{Application:} (aka function call) \code{e1 e2} means ``apply \code{e1} to \code{e2}''; \code{e1} is the \emph{function} and \code{e2} is the \code{argument}
              \end{itemize}
        \item \emph{Syntactic Sugar:} convenient notation used as a shorthand for valid syntax
              \begin{lstlisting}
-- instead of:              we write:
\x -> (\y -> (\z -> e))     \x -> \y -> \z -> e
\x -> \y -> \z -> e         \x y z -> e
(((e1 e2) e3) e4)           e1 e2 e3 e4
        \end{lstlisting}
        \item \emph{Scope of a variable} The part of a program where a \emph{variable is visible}
        \item In the expression \code{\x -> e}
              \begin{itemize}
                  \item \code{x} is the newly-introduced variable
                  \item \code{e} is the \emph{scope} of \code{x}
                  \item Any occurrence of \code{x} in \code{\x -> e} is \emph{bound} (by the \emph{binder} \code{\x})
                  \item An occurrence of \code{x} in \code{e} is \emph{free} if it is \code{not bound} by an enclosing abstraction
              \end{itemize}
        \item \emph{Free Variables:} A variable \code{x} is \emph{free} if there exists a free occurrence of \code{x} in \code{e} (not bound as a formal)
        \item \emph{Closed Expressions:} if \code{e} has no free variables it is \emph{closed}
        \item $\alpha$-step (renaming formals): we can rename a formal parameter and replace all its occurrences in the body
        \item $\beta$-step (aka function call)
              \begin{itemize}
                  \item \code{(\x -> e1) e2 =b> e1[x := e2]}
                  \item \code{e1[x := e2]} means ``\code{e1} with all free occurrences of \code{x} replaced with \code{e2}''
                  \item Computation is \code{search and replace}: if you see an \emph{abstraction} applied to an argument, take the \emph{body} of the abstraction and replace all free occurrences of the \code{formal} by that argument
              \end{itemize}
        \item \emph{Normal Forms:}
              \begin{itemize}
                  \item A \emph{redex} is a $\lambda$-term of the form \code{(\x -> e1) e2}
                  \item A $\lambda$-term is in \emph{normal form} if it contains no redexes
              \end{itemize}
        \item \emph{Evaluation:}
              \begin{itemize}
                  \item A $\lambda$-term \code{e} evaluates to \code{e'} if there is a sequence of steps
                        \begin{lstlisting}
e =?> e_1 =?> ... =?> e_N =?> e'
                \end{lstlisting}
                  \item each \code{=?>} is either \code{=a>} or \code{=b>} and \code{N >= 0}
                  \item \code{e'} is in normal form
                  \item \code{e1 =*> e2}: \code{e1} \emph{reduces} to \code{e2} in 0 or more steps
                  \item \code{e1 =~> e2}: \code{e1} \emph{evaluates} to \code{e2}
              \end{itemize}
        \item $\Omega$: \code{(\x -> x x) (\x -> x x)}
        \item \emph{Recursion:} Fixpoint Combinator
              \begin{lstlisting}
FIX STEP
=*> STEP (FIX STEP)
           \end{lstlisting}
              \begin{itemize}
                  \item \code{FIX = \stp -> (\x -> stp (x x))(\x -> stp (x x))}
              \end{itemize}
        \item \emph{Quicksort in Haskell}
              \begin{lstlisting}
sort :: [a] -> [a]
sort []     = []
sort (x:xs) = sort ls ++ [x] ++ sort rs
    where
        ls  = [ l | l <- xs, l <= x ]
        rs  = [ r | r <- xs, x < r   ]
           \end{lstlisting}
        \item \emph{Functions in Haskell}
              \begin{itemize}
                  \item Functions are \emph{first-class values}
                  \item can be \emph{passes as arguments} to other functions
                  \item can be \emph{returned as results} from other functions
                  \item can be \emph{partially applied} (arguments passed \emph{one at a time})
              \end{itemize}
        \item \emph{Top-level bindings:}
              \begin{itemize}
                  \item Things can be defined globally
                  \item Their names are called \emph{top-level variables}
                  \item Their definitions are called \emph{top-level bindings}
              \end{itemize}
        \item \emph{Equations and Patterns}
              \begin{lstlisting}
pair x y b  = if b then x else y
fst p       = p True
snd p       = p False
\end{lstlisting}
              \begin{itemize}
                  \item A single function binding can have multiple equations with different \emph{patterns} of parameters
                  \item The first equation whose pattern matches the actual arguments is chosen
              \end{itemize}
        \item \emph{Referential Transparency} means that a variable can be defined \emph{once per scope} and \emph{no mutation is allowed}; the same function always evaluates to the same value
        \item \emph{Local variables} can be defined using a \code{let} expression
              \begin{lstlisting}
sum 0 = 0
sum n = let n' = n - 1
        in n + sum n'
           \end{lstlisting}
        \item Syntactic sugar for nested \code{let} expressions:
              \begin{lstlisting}
sum 0 = 0
sum n = let
            n'      = n - 1
            sum'    = sum n'
        in n + sum'
           \end{lstlisting}
        \item If you need a variable whose scope is an equation, use the \code{where} clause instead:
              \begin{lstlisting}
cmpSquare x y   | x > z     = "bigger :)"
                | x == z    = "same :|"
                | x < z     = "smaller :("
    where z = y * y
           \end{lstlisting}
        \item \emph{Types:}
              \begin{itemize}
                  \item In Haskell every expression either \emph{has a type} or is \emph{ill-typed} and rejected at compile-time
                  \item Types can be annotated using \code{::}
                        \begin{lstlisting}
haskellIsAwesome :: Bool
haskellIsAwesome = True
               \end{lstlisting}
                  \item Functions have \emph{arrow types}
                  \item \code{\x -> e} has type \code{A -> B}
                  \item If \code{e} has type \code{B} assuming \code{x} has type \code{A}
              \end{itemize}
        \item A \emph{Combinator} is a function with \emph{no free variables}
        \item \emph{Lists:}
              \begin{itemize}
                  \item A list is either an \emph{empty list}: \code{[ ]}
                  \item Or a \emph{head element} attached to a \emph{tail list}: \code{x:xs}
                        \begin{lstlisting}
[]                  -- A list with zero elements
1:[]                -- A list with one element
(:) 1 []            -- A list with one element
1:(2:(3:(4:[])))    -- A list with four elements
1:2:3:4:[]          -- Same thing
[1,2,3,4]           -- Syntactic sugar
               \end{lstlisting}
                  \item \code{[]} and \code{:} are called the list \emph{constructors}
                  \item A list has type \code{[A]} if each one of its elements has type \code{A}
              \end{itemize}
        \item \emph{Pairs:} the constructor is \code{(,)}
              \begin{lstlisting}
myPair :: (String, Int)
myPair = ("apple", 3)
               \end{lstlisting}
        \item \emph{Record Syntax:}
              \begin{itemize}
                  \item Instead of:
                        \begin{lstlisting}
data Date = Date Int Int Int
               \end{lstlisting}
                  \item You can write:
                        \begin{lstlisting}
data Date = Date {
    month   :: Int,
    day     :: Int,
    year    :: Int
}
               \end{lstlisting}
                  \item Use the field name as a function to access part of the data:
                        \begin{lstlisting}
deadlineDate = Date 1 10 2019
deadlineMonth = month deadlineDate
               \end{lstlisting}
              \end{itemize}
        \item Building data types:
              \begin{itemize}
                  \item \emph{Product} types (each-of): a value of \code{T} contains a value of \code{T1} \emph{and} a value of \code{T2}
                  \item \emph{Sum} types (one-of): a value of \code{T} contains a value of \code{T1} \emph{or} a value of \code{T2}
                  \item \emph{Recursive} types: a value of \code{T} contains a \emph{sub-value} of the same type \code{T}
              \end{itemize}
        \item \emph{Pattern Matching:}
              \begin{lstlisting}
html :: Paragraph -> String
html (Text str)         = ...
html (Heading lvl str)  = ...
html (List ord items)   = ...
\end{lstlisting}
              \begin{itemize}
                  \item Match for arbitrary data types
                  \item Dangers: \emph{missing} or \emph{overlapped} patterns
                  \item Pattern matching expression
                        \begin{lstlisting}
html :: Paragraph -> String
html p =
    case p of
        Text str        -> ...
        Heading lvl str -> ...
        List ord items  -> ...
\end{lstlisting}
                  \item The \code{case} expression has type \code{T} if every output expression has type \code{T} and the input is a valid pattern for the type; the input expression is called the \emph{match scrutinee}
              \end{itemize}
        \item \emph{Tail Recursion:} The recursive call is the \emph{top-most} sub-expression in the function body; no computations allowed on recursively-returned body; the value returned by the recursive call is the value returned by the function
        \item Tail-recursive factorial:
              \begin{lstlisting}
loop acc n
    | n <= 1    = acc
    | otherwise = loop (acc * n) (n - 1)
              \end{lstlisting}
        \item Tail recursive calls compile to fast loops automatically
        \item The \emph{Filter} pattern:
              \begin{lstlisting}
filter :: (a -> Bool) -> [a] -> [a]
filter f []     = []
filter f (x:xs)
    | f x       = x : filter f xs
    | otherwise =     filter f xs
              \end{lstlisting}
              \begin{itemize}
                  \item Higher-order function which takes function \code{f} and a list as arg
                  \item For each element \code{x} in the list, if \code{f x == True} then \code{x} will be in the output list
              \end{itemize}
        \item The \emph{Map} pattern:
              \begin{lstlisting}
map :: (a -> b) -> [a] -> [b]
map f []        = []
map f (x:xs)    = f x : map f xs
              \end{lstlisting}
              \begin{itemize}
                  \item Higher order function which takes a function \code{f} and a list as arg
                  \item For each element \code{x} in the input list, \code{f x} will be in the output list
              \end{itemize}
        \item The \emph{Fold-Right} pattern:
              \begin{lstlisting}
foldr :: (a -> b -> b) -> b -> [a] -> b
foldr f b []        = b
foldr f b (x:xs)    = f x (foldr f b xs)
            \end{lstlisting}
              \begin{itemize}
                  \item Higher order function which recurses on the tail
                  \item Combines result with the head in some binary operation
                  \item \code{len = foldr (\x n -> 1 + n) 0}
                  \item \code{sum = foldr (\x n -> x + n) 0}
                  \item \code{cat = foldr (\x n -> x ++ n) ""}
              \end{itemize}
        \item The \emph{Fold-Left} pattern:
              \begin{lstlisting}
foldl :: (a -> b -> a) -> a -> [b] -> a
foldl f b xs                = helper b xs
    where
        helper acc []       = acc
        helper acc (x:xs)   = helper (f acc x) xs
            \end{lstlisting}
              \begin{itemize}
                  \item Higher order function uses a helper function with an extra accumulator argument
                  \item To compute the new accumulator, combine the urrent accumulator with the head using some binary operation
              \end{itemize}
        \item Useful HOFs:
              \begin{itemize}
                  \item \code{Flip:} flips the order of the input args
                        \begin{lstlisting}
flip :: (a -> b -> c) -> b -> a -> c
                \end{lstlisting}
                  \item \code{Compose:} compose functions
                        \begin{lstlisting}
(.) :: (b -> c) -> (a -> b) -> a -> c
                \end{lstlisting}
              \end{itemize}
        \item Libraries will implement \code{map}, \code{fold}, \code{filter}, etc on its collections
        \item \emph{List Comprehensions:} List comprehensions construct a new list from an old list
              \begin{lstlisting}
-- examples
map f xs    = [f x | x <- xs]
filter p xs = [x | x <- xs, p x]
concat xss  = [x | xs <- xss, x <- xs]

-- rules
[e |True]           = [e]
[e | q]             = [e | q, True]
[e | b, Q]          = if b then [e | Q] else []
[e | p <- xs, Q]    = let ok p = [e | Q]
                          ok _ = []
                      in concat (map ok xs)
            \end{lstlisting}
            \item \emph{Functions from Homework:}
\begin{lstlisting}
sumList :: [Int] -> Int
sumList []      = 0
sumList (x:xs)  = x + sumList xs
\end{lstlisting}

\begin{lstlisting}
digitsOfInt :: Int -> [Int]
digitsOfInt n
    | n < 0     = []
    | n < 10    = [n]
    | otherwise = digitsOfInt (div n 10) ++ [mod n 10]

digits :: Int -> [Int]
digits n = digitsOfInt (abs n)
\end{lstlisting}
\begin{lstlisting}
additivePersistence :: Int -> Int
additivePersistence n~
    | n < 10    = 0
    | otherwise = 1 + additivePersistence (sumList (digitsOfInt n))
\end{lstlisting}
\begin{lstlisting}
digitalRoot :: Int -> Int
digitalRoot n
    | n < 10    = n
    | otherwise = digitalRoot (sumList (digitsOfInt n))
\end{lstlisting}
\begin{lstlisting}
listReverse :: [a] -> [a]
listReverse []      = []
listReverse (x:xs)  = listReverse xs ++ [x]
\end{lstlisting}
    \end{itemize}
    \begin{center}
        \begin{adjustbox}{angle=270}
            \begin{lstlisting}[caption="Lambda Calculus Functions" label=""]
let ZERO    = \f x -> x
let ONE     = \f x -> f x
let TWO     = \f x -> f (f x)
let SUCC    = \n f x -> f (n f x)
let EXP     = \n m -> n (MULT m) ONE

let TRUE    = \x y -> x
let FALSE   = \x y -> y
let ITE     = \b x y -> b x y
let AND     = \b1 b2 -> ITE b1 b2 FALSE
let OR      = \b1 b2 -> ITE b1 TRUE b2
let NOT     = \b1 -> b1 FALSE TRUE
let NOR     = \b1 b2 -> NOT (OR b1 b2)
let NAND    = \b1 b2 -> NOT (AND b1 b2)
let XOR     = \b1 b2 -> AND (NAND b1 b2) (OR b1 b2)

let INCR    = \n f x -> f (n f x)

let PAIR    = \x y b -> b x y
let FST     = \p     -> p TRUE
let SND     = \p     -> p FALSE

let SKIP1   = \j k -> (\b -> b TRUE ((AND TRUE (k(TRUE))) (j(k(FALSE))) (k(FALSE))))
let DECR    =  \n -> (n (SKIP1 INCR) (PAIR FALSE ZERO)) FALSE
let SUB     = \m n -> (n DECR) m
let ISZ     =  \n -> n(\a -> FALSE) TRUE
let EQL     = \n m -> AND (ISZ (SUB m n)) (ISZ (SUB n m))
let SUC   = \n f x -> f (n f x)
let ADD     = \n m -> n SUC m
let MUL     = \n m -> n (ADD m) ZERO
let REPEAT  = \n m -> n (PAIR m) FALSE
let EMPTY   = \p -> p (\x y z -> FALSE) TRUE
let FIX     = \stp -> (\x -> stp (x x)) (\x -> stp (x x))
let LEN     = FIX (\rec n -> (EMPTY n) ZERO (INCR (rec (SND n))))
let STEP    = \rec n -> ITE (ISZ n) ZERO (ADD n (rec (DECR n)))
let SUM     = FIX STEP
let DIV     = FIX (\rec n m -> ITE (EQL n m) ONE (ITE (ISZ (SUB n m) ) ZERO (INCR (rec (SUB n m) m))))
let MOD     = FIX (\rec n m -> ITE (EQL n m) ZERO (ITE (ISZ (SUB n m) ) n (rec (SUB n m) m)))
let INSERT  = \n m -> (PAIR n m)
let APPEND' = FIX (\rec n m -> ITE (EMPTY n) m (INSERT (FST n) (rec (SND n) m)))
            \end{lstlisting}


        \end{adjustbox}
    \end{center}
\end{multicols}

\end{document}
