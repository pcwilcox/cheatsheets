% LaTeX template adapted from: https://www.overleaf.com/latex/templates/simple-math-homework-template/tbszsswsndrz
\documentclass[landscape,10pt]{article}
\usepackage[utf8]{inputenc}
\usepackage[english]{babel}
\usepackage[]{amsthm} %lets us use \begin{proof}
\usepackage[]{amssymb} %gives us the character \varnothing
\usepackage{amsmath} %for equations
\usepackage[]{listings} %for code blocks
\usepackage{graphicx} %for diagrams
\usepackage{fancyhdr} %for headers
\usepackage[letterpaper, margin=0.5in]{geometry}
\usepackage{tikz} % for drawings
\usepackage{multicol}
\usepackage{ifthen}
\usepackage{enumitem}
\setlist{nolistsep}
\setlist[itemize]{leftmargin=0.5pc,itemsep=0.25em}
\usetikzlibrary{arrows.meta,shapes.arrows,chains,decorations.pathreplacing}
\makeatletter
\renewcommand{\section}{\@startsection{section}{1}{0mm}%
            {-1ex plus -.5ex minus -.2ex}%
            {0.5ex plus .2ex}%x
            {\normalfont\large\bfseries}}
\renewcommand{\subsection}{\@startsection{subsection}{2}{0mm}%
            {-1explus -.5ex minus -.2ex}%
            {0.5ex plus .2ex}%
            {\normalfont\normalsize\bfseries}}
\renewcommand{\subsubsection}{\@startsection{subsubsection}{3}{0mm}%
            {-1ex plus -.5ex minus -.2ex}%
            {1ex plus .2ex}%
            {\normalfont\small\bfseries}}
\makeatother

\pagestyle{empty}
\ifthenelse{\lengthtest { \paperwidth = 11in}}
{ \geometry{top=.5in,left=.5in,right=.5in,bottom=.5in} }
{\ifthenelse{ \lengthtest{ \paperwidth = 297mm}}
{\geometry{top=1cm,left=1cm,right=1cm,bottom=1cm} }
{\geometry{top=1cm,left=1cm,right=1cm,bottom=1cm} }
}
\setlength{\parindent}{0em}
\setlength{\parskip}{-0.25em}

\begin{document}
\footnotesize
\begin{multicols}{2}
\setlength{\premulticols}{1pt}
\setlength{\postmulticols}{1pt}
\setlength{\multicolsep}{1pt}
\setlength{\columnsep}{2pt}
\subsubsection*{Some Formulae:}
\begin{itemize}
    \item[] \(\text{Power} = \text{Capacity} \cdot \text{Voltage}^2 \cdot \text{Frequency}_{0 \rightarrow 1} + \text{Voltage}\cdot I_{\text{leakage}}\)
    \item[] \(\text{Power} = \frac{\text{Joules}}{\text{Op}} \cdot \frac{\text{Ops}}{\text{Second}}\)
    \item[] Dennard scaling: 0.7x voltage drop 
    \item[] Latency: How long it takes to do a task
    \item[] Throughput: Total work done per unit of time e.g. queries/sec
    \item[] Clock period: Duration of a clock cycle
    \item[] Clock frequency: cycles per second
    \item[] \(\text{Execution Time} = \text{Cycles per Program} \cdot \text{Clock Cycle Time} = \frac{\text{Cycles Per Program}}{\text{Clock Rate}}\)
    \item[] \(\text{Clock Cycles} = \text{Instruction Count} \cdot \text{Cycles per Instruction}\)
    \item[] \(\text{CPU Time} = \text{Instruction Count} \cdot \text{CPI} \cdot \text{Clock Cycle Time} = \frac{IC \cdot CPI}{\text{Clock Rate}}\)
    \item[] \(\text{Exec Time} = \frac{\text{Instructions}}{\text{Program}} \cdot \frac{\text{Clock Cycles}}{\text{Instruction}} \cdot \frac{\text{Seconds}}{\text{Clock Cycle}}\)
    \item[] Clock Cycles \(= \sum \limits_{i = 1}^{n} \big ( CPI_i \cdot IC_i \big) \)
    \item[] Weighted average CPI \(= \frac{\text{Clock Cycles}}{\text{Instruction Count}} = \sum \limits_{i=1}^{n} \Big ( CPI_i \cdot \frac {IC_i}{IC} \Big ) \)
    \item[] Performance \(= \frac{1}{\text{Exec Time}}\)
    \item[] ``\(X\) is \(n\) faster than \(Y\)'' means \(\frac{Perf_X}{Perf_Y} = \frac{Exec_Y}{Exec_X} = n\)
    \item[] Energy = Average Power x Execution Time
    \item[] Optimize for Energy per Instruction: \(Power = \frac{energy}{second} = \frac{energy}{instruction} \cdot \frac{instructions}{second}\)
    \item[] Amdahl's Law: \(Speedup = \frac{CPUTime_{old}}{CPUTime_{new}} = \frac{CPUTime_{old}}{CPUTime_{old} [(1 - f_x) + \tfrac{f_x}{S_x}]} = \frac{1}{(1 - f_x) + \tfrac{f_x}{s_x}}\)
    \item[] Parallel Speedup: \(Speedup = \frac{1}{(1 - P) + \tfrac{P}{n}} \rightarrow P = \frac{\frac{1}{Speedup} - 1}{\tfrac{1}{n} - 1}\)
    \item[] Arithmetic Mean: \(\frac{1}{n} \sum \limits_{i=1}^{n}T_i\) used with times, not rates
    \item[] Harmonic mean: \(\frac{n}{\sum \limits_{i=1}^{n}\frac{1}{R_i}}\) used with rates not times
    \item[] Geometric mean: \(\Bigg ( \prod \limits_{i=1}^{n} \frac{T_i}{T_{ri}} \Bigg)^{\tfrac{1}{n}} = \exp\Bigg(\frac{1}{n}\sum \limits_{i=1}^{n}\log\Big(\frac{T_i}{T_{ri}}\Big) \Bigg)\)
    \item[] Power/performance benchmark: Overall ssj\_ops per Watt = \(\frac{\Big(\sum \limits_{i=0}^{10}ssj\_ops_i\Big)}{\Big(\sum\limits_{i=0}^{10}Power_i\Big)}\)
    \item[] Perf metrics: 
    \begin{itemize}
        \item[] \texttt{time <application>} measures execution time
        \item[] \texttt{perf record} does low overhead sampling
        \item[] \texttt{perf topdown} uses hardware performance counters
    \end{itemize}
    \item[] Flip flop setup time: \(t_{ck} > t_{pd} + t_s + t_{skew}\)
\end{itemize}

\subsubsection*{CISC:}
\begin{itemize}
    \item[] Multi-cycle complex instructions
    \item[] Load/store incorporated in instruction
    \item[] Small code size
    \item[] High CPI
    \item[] Low clock frequency
    \item[] Variable length instructions
\end{itemize}
\subsubsection*{RISC:}
\begin{itemize}
    \item[] Simple (single-clock) Instructions
    \item[] Register-to-register separate load instructions
    \item[] Large code size
    \item[] Low CPI
    \item[] High clock frequency
    \item[] Same length instructions
    \item[] Simple instruction decode
    \item[] Instructions:
    \begin{itemize}
        \item[] R-type: \texttt{[funct7][rs2][rs1][funct3][rd][opcode]}
        \item[] I-type: \texttt{[immediate[11:0]][rs1][funct3][rd][opcode]}
        \item[] S/B-type: \texttt{[imm[11:5]][rs2][rs1][funct3][imm[4:0]][opcode]}
        \item[] U/J-type: \texttt{[immediate[31:12]][rd][opcode]}
    \end{itemize}
\end{itemize}

\subsubsection*{Call and Return:}
\begin{itemize}
    \item[] Caller:
    \begin{itemize}
        \item[] Save caller-saved registers as needed
        \item[] Load arguments
        \item[] Execute \texttt{JAL}
    \end{itemize}
    \item[] Callee setup:
    \begin{itemize}
        \item[] Allocate memory for new frame (\texttt{xsp} = \texttt{xsp} - frame)
        \item[] Save callee-saved registers as needed
        \item[] Set frame pointer (\texttt{xfp} = \texttt{xsp} + frame size - 4)
    \end{itemize}
    \item[] Callee return:
    \begin{itemize}
        \item[] Place return values in \texttt{x10} and \texttt{x11}
        \item[] Restore any callee-saved registers
        \item[] Pop stack (\texttt{xsp} = \texttt{xsp} + frame size)
        \item[] Return by \texttt{jr xra}
    \end{itemize}
    \item[] Caller:
    \begin{itemize}
        \item[] Restore any caller-saved registers as needed
    \end{itemize}
\end{itemize}
\subsubsection*{Pipeline stages:}
\begin{itemize}
    \item[] IF: Instruction fetch
    \item[] ID: Instruction decode, register read
    \item[] EX: execute operation or calculate address
    \item[] MEM: Access memory command
    \item[] WB: Write result back to register
\end{itemize}
\subsubsection*{Hazards:}
\begin{itemize}
    \item[] Structure Hazards: A required hardware resource is busy
    \item[] Data hazards: Must wait for previous instructions to produce/consume data
    \begin{itemize}
        \item[] Read after Write: Instruction \(j\) tries to read before instruction \(i\) tries to write it
        \item[] Write after Write: Instruction \(j\) tries to write an operand before \(i\) writes its value
        \item[] Write after Read: Instruction \(j\) tries to write a destination before it is read by \(i\)
    \end{itemize}
    \item[] Control hazards: Next PC depends on current instruction result
\end{itemize}
\subsubsection*{Forwarding:}
\begin{itemize}
    \item[] Identify \textit{producers}: EX and MEM stages
    \item[] All stages after first producer are sources of forwarding data: MEM and WB
    \item[] Identify \textit{consumers}: EX and MEM
    \item[] These are the destinations of forwarded data
    \item[] Forwarding and Hazard Detection:
    \begin{verbatim}
if (MEM/WB.RegWrite && (MEM/WB.RegisterRd != 0) && 
    !(EX/MEM.RegWrite && (EX/MEM.RegisterRd != 0) && 
    (EX/MEM.RegisterRd = ID/EX.RegisterRs1)) && (MEM/WB.RegisterRd = ID/EX.RegisterRs1)) 
        ForwardA = 01
if (MEM/WB.RegWrite && (MEM/WB.RegisterRd != 0) && 
    !(EX/MEM.RegWrite && 
    (EX/MEM.RegisterRd != 0) && (EX/MEM.RegisterRd = ID/EX.RegisterRs2)) && 
    (MEM/WB.RegisterRd = ID/EX.RegisterRs2)) 
        ForwardB = 01
if (ID/EX.MemRead && ((ID/EX.RegisterRd = IF/ID.RegisterRs1) || 
    (ID/EX.RegisterRd = IF/ID.RegisterRs2))) 
        stall the pipeline
    \end{verbatim}
\end{itemize}
\subsubsection*{Branch Prediction:}
\begin{itemize}
    \item[] Static: predict not-taken
    \item[] Dynamic:
    \begin{itemize}
        \item[] Branch History Table: one entry for each branch, taken/not taken record
        \item[] Branch Target Buffer: One entry for each branch, computes target address
    \end{itemize}
\end{itemize}
\subsubsection*{Exceptions and Interrupts:}
\begin{itemize}
    \item[] Switch from `user' to `kernel' mode
    \item[] Exceptions: Arises within CPU. Undefined opcode, overflow, etc
    \item[] Interrupt: External I/O controller, network card, etc
    \item[] On exceptions:
    \begin{itemize}
        \item[] Pass to relevant handler
        \item[] Then return to program using EPC if possible
        \item[] All previous instructions completed
        \item[] Faulting instruction not started
        \item[] No side effects
    \end{itemize}
    \item[] Nullifying instructions: Converts them to a NOP
\end{itemize}
\subsubsection*{Going past the 5-stage pipeline:}
\begin{itemize}
    \item[] Instruction-level Parallelism:
    \begin{itemize}
        \item[] Independence among instructions
        \item[] Fetch multiple instructions per cycle
        \item[] Evaluate which ones can go down the pipe together
    \end{itemize}
    \item[] Superscalar:
    \begin{itemize}
        \item[] Double up on hardware in order to execute multiple instructions simultaneously
    \end{itemize}
    \item[] Deeper Pipelines: Increase the number of stages, decrease amount of logic. Higher clock freq.
    \item[] Register Renaming: Map architectural registers to physical registers in decode stage to get rid of false dependencies. Need more physical registers than architectural ones.
    \item[] Data flow graph:
    \begin{itemize}
        \item[] Scoreboard: bit array, 1-bit for each GPR
        \begin{itemize}
            \item[] If the bit is not set, the reg has valid data
            \item[] If the bit is set, the reg has stale data
        \end{itemize}
        \item[] Dispatch in order: RD \(\leftarrow\) Fn(RS,RT)
        \begin{itemize}
            \item[] If SB[RS] or SB[RT] is set \(\rightarrow\) RAW, stall
            \item[] If SB[RD] is set \(\rightarrow\) WAW, stall
            \item[] Else dispatch to functional unit, set SB[RD]
        \end{itemize}
        \item[] Complete out-of-order
        \item[] Update GPR[RD], clear SB[RD]
    \end{itemize}
\end{itemize}

\subsubsection*{Caching:}
\begin{itemize}
    \item[] AMAT: \(\text{Access Time} = \text{hit time} + \text{miss rate} \cdot \text{miss penalty}\)
    \item[] Three C's of misses:
    \begin{itemize}
        \item[] Compulsory: First time you've accessed this item
        \item[] Capacity: Not enough room in the cache to hold item
        \item[] Conflict: Item was replaced because of a conflict in its set
    \end{itemize}
    \item[] Direct-mapped cache:
    \begin{itemize}
        \item[] \(2^n\) bytes total with \(2^m\) byte blocks
        \item[] Byte select: lower \(m\) bits
        \item[] Cache index: lower \((n-m)\) bits of the memory address
        \item[] Cache tag: upper \(32-n\) bits of the memory address
    \end{itemize}
    \item[] \(N\)-way set associative:
    \begin{itemize}
        \item[] Each memory block can go to one of \(N\) entries in the cache
        \item[] \(2^n\)-byte cache, \(2^m\)-byte blocks, \(2^a\) set-associative:
        \begin{itemize}
            \item[] Cache contains \(2^n/2^m = 2^{n-m}\) blocks
            \item[] Each cache way contains \(2^{n-m}/2^a = 2^{n-m-a}\) blocks
            \item[] Byte offset: lowest \(m\) bits
            \item[] Cache index: next \(n-a\) bits
        \end{itemize}
        \item[] Associative caches might use Least Recently Used table for evictions
        \item[] For \(N\)-way cache, \(N!\) orderings
    \end{itemize}
    \item[] Write-through:
    \begin{itemize}
        \item[] Main memory updated each cache write
        \item[] Replacing a cache entry just overwrites new block
        \item[] Memory write may cause pipeline stalls
        \item[] Misses are simpler and cheaper
        \item[] Uses a write buffer | FIFO queue
    \end{itemize}
    \item[] Write-back:
    \begin{itemize}
        \item[] Only the cache entry is updated on each cache write
        \item[] Cache and memory entries are inconsistent
        \item[] Add `dirty' bit to indicate whether memory needs to be updated
        \item[] Write new value to memory on evictions
        \item[] Writes are super fast
    \end{itemize}
    \item[] Write miss options: Do you allocate for space in the cache on a miss?
    \item[] Do you fetch the rest of the block contents from memory?
    \item[] For no-fetch-on-miss must use fine-grained valid bits
    \item[] Write-back: typically write-allocate, fetch-on-miss
    \item[] Multilevel caches:
    \begin{itemize}
        \item[] Primary L1 caches attached to CPU: small, fast, focuses on hit time
        \item[] Secondary L2 caches service misses from L1: larger, slower, still faster than DRAM
    \end{itemize}
    \item[] Cache Coherency:
    \begin{itemize}
        \item[] \(P\) writes \(X\), \(P\) reads \(X \rightarrow \) read returns written value
        \item[] \(P_1\) writes \(X\), \(P_2\) reads \(X\) later \(\rightarrow \) read returns written value
        \item[] \(P_1\) writes \(X\), \(P_2\) writes \(X \rightarrow\) all processors see writes in the same order
        \item[] Single-Writer, Multiple-Read Invariant: For any memory location \(A\), at any given epoch, there exists only one CPU that may write to \(A\) or some number of CPUs that may only read \(A\)
        \item[] Data-Value Invariant: The value of the memory location at the start of an epoch is the same as the value of the memory location at the end of its last read-write epoch
        \item[] CPU uses snooping protocols to ensure this:
        \begin{itemize}
            \item[] 2-state: very simple hardware and protocol. Write-through, all writes go on interconnect bus.
            \item[] 3-state MSI: Modified (one cache has valid copy), Shared (one or more have read-only copy), Invalid (invalidated so one copy can go to modify state)
        \end{itemize}

    \end{itemize}
\end{itemize}


    \end{multicols}
\end{document}

