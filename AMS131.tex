% LaTeX template adapted from: https://www.overleaf.com/latex/templates/simple-math-homework-template/tbszsswsndrz
\documentclass{article}
\usepackage[utf8]{inputenc}
\usepackage[english]{babel}
\usepackage[]{amsthm} %lets us use \begin{proof}
\usepackage[]{amssymb} %gives us the character \varnothing
\usepackage{amsmath} %for equations
\usepackage[]{listings} %for code blocks
\usepackage{graphicx} %for diagrams
\usepackage{fancyhdr} %for headers
\usepackage[letterpaper, margin=1in]{geometry}
\usepackage{tikz} % for drawings
\usepackage{multicol}
\usetikzlibrary{arrows.meta,shapes.arrows,chains,decorations.pathreplacing}


\graphicspath{}
\pagestyle{fancy}
\setlength{\parindent}{1em}
\setlength{\parskip}{0em}
\rhead{Pete Wilcox | pcwilcox@ucsc.edu}
\lhead{AMS 131: Cheat Sheet}  

\begin{document}
    \subsection*{Experiments and Events:}
        \paragraph*{Def:} An experiment is a process whose outcome is not known in advance with certainy.
        \paragraph*{Sample Space: } Collection of \textit{all} possible outcomes of an experiment. $S$ or $\Omega$. Each outcome is an element of the sample space $s \in S$.
    \subsection*{Operations: }
        \paragraph*{Union: } \( x \in S: A \cup B = \{x \in A \text{ or } x \in B\}\)
            \begin{itemize}
                \item[] \(A \cup B = B \cup A \)
                \item[] \(A \cup A = A \)
                \item[] \(A \cup \emptyset = A \)
                \item[] \(A \cup S = S \)
                \item[] \(A \subset B \Rightarrow A \cup B = B\)
                \item[] \(A_1, A_2, \dots, A_n \Rightarrow A_1 \cup A_2 \cup \dots \cup A_n = \bigcup\limits_{i=1}^{i=n} A_i\)
                \item[] \(\bigcup\limits_{i=1}^{\infty} A_i \rightarrow \bigcup\limits_{i\in I}A_i \)
                \item[] \( (A \cup B) \cup C = A \cup (B \cup C) = A \cup B \cup C \)
            \end{itemize}
        \paragraph*{Intersections: } \( A \cap B = \{ x \in A \text{ and } x \in B\} = AB \)            

            

\end{document}